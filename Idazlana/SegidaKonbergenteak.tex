\begin{document}


\begin{titlepage}

\begin{center}
\vspace*{-1in}
\begin{figure}[htb]
\begin{center}
\includegraphics[width=8cm]{./figuras/logo}
\end{center}
\end{figure}


ANALISI MATEMATIKOA\\
\vspace*{1.15in}
\begin{Huge}
\textbf{SEGIDAK. SEGIDA KONBERGENTEA}\\
\end{Huge}
\vspace*{0.6in}
\begin{large}
2014-2015 IKASTURTEA\\
\vspace*{0.15in}
(EUSKERA TALDEA)\\
\end{large}
\vspace*{1.2in}
\begin{Large}
\textbf{Informatika Ingeniaritzako Gradua} \\
\end{Large}
\begin{Large}
\textbf{Informatika Fakultatea} \\
\end{Large}
\begin{Large}
\textbf{Donostia} \\
\end{Large}
\vspace*{2.3in}
\vspace*{0.3in}
\rule{80mm}{0.1mm}\\
\vspace*{0.1in}
\begin{large}
Ander Lopez \\
\end{large}
\end{center}

\end{titlepage}


\section{Segida Konbergentea}

\begin{propietate}
\item

$\{a_{n} \}$ segida konbergentea bada, limite bakarra du.\\

\underline{\bold{Froga:}}\\

Demagun ${l_{1}}$ eta ${l_{2}}$ direla  ${a_{n} }$ segidaren bi limite desberdin, ${l_{1}}\neq {l_{2}}$ beraz, $\mid{l_{1}}-{l_{2}}\mid> 0$ da.

Har dezagun $\varepsilon_{0}=\dfrac{\mid l_{1}- l_{2}\mid}{3}$ erradioa.
\vspace*{0.01in}
\begin{figure}[hbtp]
\caption{}
\centering
\includegraphics[scale=1]{../../../Desktop/2grafikoa.jpg}
\end{figure}

\vspace*{0.01in}
Osa ditzagun $\varepsilon((l_{1},\varepsilon_{0})$ eta $\varepsilon ({l_{2}},\varepsilon_{0})$ inguruneak. Ingurune horiek disjuntuak dira, hau da, $\varepsilon (${l_{1}}$,\varepsilon_{0})\cap \varepsilon (${l_{2}}$, \varepsilon_{0})=\emptyset$\\

-${l_{1}}$ bada ${a_{n} }$ segidaren limitea, $\varepsilon_{0}>0$ emanik, $\exists $n$ (\varepsilon_{0}) \in \NN \ / \ \forall n\geqslant n(\varepsilon_{0})$ \hspace{0.25cm} ${a_{n} }\in \varepsilon (l_{1},\varepsilon_{0})$\\

-${l_{2}}$ bada ${a_{n} }$ segidaren limitea, $\varepsilon_{0}>0$\ emanik,\\

$\exists $n_{2}$(\varepsilon_{0}) \in \NN\ / \ \forall n\geqslant n_{2}(\varepsilon_{0})$\hspace{2cm} $a_{n}   \in \varepsilon (l_{2},\varepsilon_{0})$\\
 
Hortik, $n_{0} (\varepsilon_{0})=max\ \lbrace n_{1}(\varepsilon_{0}), n_{2}(\varepsilon_{0})\rbrace$ \ bada, $\forall n\geqslant n_{0}(\varepsilon_{0})$\\
 ${a_{n} }\in \varepsilon(${l_{1}}$, \varepsilon_{0})\cap \varepsilon(${l_{2}}$, \varepsilon_{0})$\ ateratzen\ dugu.\\

\\Hori ezinezkoa da inguruneak disjuntuak direlako. Ondorioz, bi limiteak ezin dira ezberdinak izan.\\
\begin{figure}[hbtp]
\caption{}
\centering
\includegraphics[scale=1]{../../../Desktop/3grafikoa.jpg}
\end{figure}


\end{propietate}
\vspace*{0.30in}
\begin{propietate}
\item
$\{a_{n} \}$ segida konbergetea bada, bere  azpisegida guztiak konbergenteak dira eta limite bera dute.\\
\end{propietate}

\begin{propietate}
\item
${a_{n} }$ segida konbergentea bada, segida bornatua da.\\ 

\underline{\bold{Froga:}} Demagun $l$ dela ${a_{n} }$ segidaren limitea\\

$\forall \varepsilon > 0 \hspace{0.75cm}\exists n_{0}(\varepsilon)\in \NN \ / \forall n\geqslant n_{0}(\varepsilon)$\hspace{0.75cm}\underline{${a_{n} } \in \varepsilon(l,\varepsilon)$}\  edo  \underline{${a_{n} }\in(l-\varepsilon, l+\varepsilon)$}\\

\vspace*{0.10in}

\centering{edo \ \underline{$(l-\varepsilon)< \ {a_{n} } \ <(l+\varepsilon)$}}

\vspace*{0.15in}

Horrek esan nahi du segidaren $\lbrace {a_{n}_{0}}, {a_{n}_{0+1}},... \rbrace$ azpimultzoa bornaturik dagoela. Bestalde, $\lbrace {a_{1} },{a_{2} },...,{a_{n}_{0-1}} \rbrace$ azpimultzoa finitua da, beraz bornatua da.
Ondorioz, $\lbrace{a_{n} }\rbrace$ segida bi azpimultzo bornatuan banatu dugu, hau da,$\lbrace{a_{n} }\rbrace$ segida ere bornatua da.\
\end{propietate}

\begin{propietate}
\item
$\lbrace{a_{n} }\rbrace$ segida konbergentearen limitea ez bada zero, segidaren gai batetik aurrera segidaren gai guztiak limitearen zeinua dute.\\
\end{propietate}
\begin{propietate}
\item
$\lbrace{a_{n} }\rbrace&$ eta $\lbrace{b_{n} }\rbrace$ segida limite $l$ bada eta $\forall n\geqslant n_{0}$ \ ${a_{n} }\leqslant{C_{n} }\leqslant{b_{n} }$ bada,$\lbrace{C_{n} }\rbrace$ segida ere konbergentea da eta bere limitea $l$ da.\\
\end{propietate}
\begin{adibide}
\item
$\lbrace sin n \rceil$ aztertuko dugu:\

$\lbrace {D_{n}} \rbrace$ = $\lbrace sin n \ / \ sin n>\dfrac{1}{2}\rbrace$ = $\lbrace 1, sin 2, sin 7,...\rbrace$ konbergentea balitz, $\dfrac{1}{2} < l_{1} <$ 1 litzateke.\\
$\lbrace {E_{n}} \rbrace$ = $\lbrace sin n \ / \ sin n<\dfrac{-1}{2}\rbrace$ = $\lbrace 4, sin 5, sin 10,...\rbrace$ konbergentea balitz, $\dfrac{-1}{2} > l_{2} >$ -1 litzateke.\\
Hortaz, ${l_{1}}$\neq${l_{2}}$ lirateke. Ondorioz $\lbrace sin n \rbrace$ ezin da konbergentea izan.\\
Dibergentea izateko $\forall k>0$ \hspace{0.3cm} $\exists n_{0}(k)\in \NN \ / \ \forall n> n_{0}(k)$ \hspace{1cm} ${a_{n} }\in \ ext (\varepsilon(0,k)) \mid{a_{n}}\mid >k$\

\begin{figure}[hbtp]
\caption{}
\centering
\includegraphics[scale=1]{../../../Desktop/4grafiko.jpg}
\end{figure}


baina $\forall n \ sin n \in(-2,2)$, beraz, ezin da dibergentea izan.\\

Ondorioz $\lbrace sin n \rbrace$ oszilatzailea da edo $\not\exists \displaystyle{\lim_{n\to\infty} sin n}$\
\end{adibide}

\begin{adibide}
\item
$a_{n}=\dfrac{n+2}{2n-1}$\\

3, 4/3, 1, 6/7,...\\

Seguida geroz eta txikiagoa da\\

$a_{n}-a_{n+1}>0$\\

$\dfrac{n+2}{2n-1}-\dfrac{(n+1)+2}{2(n+1)-1}>0$\\

$\dfrac{n+2}{2n-1}-\dfrac{n+3}{2n+1}>0$\\

$\dfrac{2n^{2}+n+4n+2-2n^{2}-6n+n+3}{4n^{2}-1}>0$\\

$\dfrac{5}{4n^{2}-1}>0$\\

Monotonoa da:\\

-$a_{1}=3$\\

-$a_{3}=1$\\

-$a_{1000}=0.501....$\\

Beraz limitea 0,5 da. Eta konbergentea dela ere esan genezake:\\
$$0,5 < a_{n} \leqslant 3$$\\
\end{adibide}

\begin{adibide}
\item
$a_{n+1}= \sqrt[3]{4+(a_{n})^{2}}$   $n \geqslant 2$\\
$a_{1}$ izanik segida konbergentea alda?\\

$n=1$, $a_{1}=1\leqslant 2$\\
$a_{n+1}\leqslant 2$ dela konprobatu behar dugu.\\

$(a_{n})^{2} \leqslant 4\Rightarrow 4+(a_{n})^{2}\leqslant 4+4= 8\Rightarrow \sqrt[3]{4+(a_{n})^{2}} \leqslant \sqrt[3]{8} \Rightarrow a_{n+1}= \sqrt[3]{4+(a_{n})^{2}}\leqslant 2$\\

Orain monotonoa dela ikusiko dugu.\\

$a_{n+1}\geqslant a_{n}\Leftrightarrow \sqrt[3]{4+(a_{n})^{2}}\geqslant a_{n} \Leftrightarrow 4+ (a_{n})^{2}\geqslant (a_{n})^{3}\Leftrightarrow (a_{n})^{3}- (a_{n})^{2}-4\leqslant 0\Leftrightarrow (a_{n}-2)((a_{n})^{2}+(a_{n})+2)\leqslant 0$\\

Lehen $0\leqslant a_{n}\leqslant 2$ dela frogatu dugu beraz $a_{n+1}\geqslant a_{n}$ betetzen da.\\

Beraz Segida konbergentea da.\\

\begin{figure}[hbtp]
\caption{}
\centering
\includegraphics[scale=1]{../../../Desktop/5grafriko.jpg}
\end{figure}

\end{adibide}

\begin{adibide}
\item
$\{a_{n} \}_{n=1}^{\infty}$ segida, segida konbergentea alda? Eta bere limitea?\\

Konbergentea izateko monotonoa eta bornatua izan behar du.\\

n ri balioak emanik ikus daiteke monotonoa dela beraz bornatua legoke.\\

Proba dezagun $0< a_{n}\leqslant 1$.\\

$0< a_{n}\leqslant 1$ bada orduan $0<a_{n+1}\leqslant 1$ orduan suposatuz $0< a_{n}\leqslant 1$ hau ere egia da $$-1\leqslant -a_{n}< 0$$\\

Biei 3a gehituz:\\

$2< 3-a_{n}< 3\Leftrightarrow \dfrac{1}{3}< \dfrac{1}{3-a_{n}}< \dfrac{1}{2}$\\

Orduan, $0< a_{n+1}\leqslant 1$\\

Monotonoa alda?\\

$a_{n+1}\leqslant a_{n}$:\\

$\dfrac{1}{3-a_{n}}\leqslant a_{n}\Leftrightarrow 1\leqslant (3-a_{n})a_{n}\Leftrightarrow (a_{n})^{2}-3a_{n}+1\leqslant 0$\\

Beraz bai betetzen da eta beraz konbergentea da.\\

Limitea kalkulatuko dugu orain.\\
L deituko diogu $a_{n}$ segidari:\\

$\displaystyle{\lim_{n\to\infty}a_{n}= \displaystyle{\lim_{n\to\infty}{\dfrac{1}{3-a_{n-1}}}={\dfrac{1}{3-\displaystyle{\lim_{n\to\infty}{a_{n-1}}}}}$\\

L-k hau bete behar du:\\

$L=\frac{1}{3-L}\Leftrightarrow L(3-L)=1$\\

$x(3-x)=1\Leftrightarrow x^{2}-3x+1=0$\\

$x=\dfrac{3\pm\sqrt{9-4}}{2}=\dfrac{3\pm\sqrt{5}}{2}$\\
Beraz $L=\dfrac{3-\sqrt{5}}{2}< 1$\\

\begin{figure}[hbtp]
\caption{}
\centering
\includegraphics[scale=1]{../../../Desktop/6grafikoa.jpg}
\end{figure}

\end{adibide}
\vspace*{0.60in}
\subsection{Segida Monotonoa}
\begin{definizio}
\item
${a_{n}} \subset \RR$ segida emanik,
\begin{itemize}
\item
${a_{n}}$ monotono gorakorra da $\forall n \geqslant {n_{0}}$ \hspace{0.3cm} ${a_{n}} \leqslant {a_{n+1}}$ betetzen bada.
\item
${a_{n}}$ hertsiki monotono gorakorra da $\forall n \geqslant {n_{0}}$ \hspace{0.3cm} ${a_{n}}<{a_{n+1}}$ betetzen bada.
\item
${a_{n}}$ monotono beherakorra da $\forall n \geqslant {n_{0}}$ \hspace{0.3cm} ${a_{n+1}} \leqslant {a_{n}}$ betetzen bada.
\item
${a_{n}}$ hertsiki monotono beherakorra da $\forall n \geqslant {n_{0}}$ \hspace{0.3cm} ${a_{n+1}}<{a_{n}}$ betetzen bada.
\end{itemize}
\end{definizio}
\begin{adibide}
\item
\begin{itemize}
\item
$\lbrace 1 \rbrace$ segida monotono gorakorra eta beherakorra da.
\item
$\lbrace \dfrac{n^{2}-1}{n^{2}} \rbrace$ segida hertsiki monotono gorakorra da.
\item
$\lbrace \dfrac{n^{2}+1}{n^{2}} \rbrace$ segida hertsiki monotono beherakorra da.
\end{itemize}
\end{adibide}
\begin{teorema}
\item
Segida monotono bornatuak konbergenteak dira.
\end{teorema}
\begin{adibide}
\item
$\lbrace (1+\dfrac{1}{n})^{n} \rbrace \rightarrow e$\\

Froga daiteke edozein $n$ hartuta dagoela $2\leqslant(1+\dfrac{1}{n})^{n}<3$, hau da, bornatua dela. Segida hertsiki monotono gorakorra da.
\end{adibide}

\begin{adibide}
\item

$a_{n}=\dfrac{2n-1}{n}$ segida monotonoa alda?\\

Ikus dezagun hau betetzen duen:\\

$a_{n+1}-a_{n} > 0   \forall n\in \NN$\\

$a_{n+1}-a_{n}=\dfrac{2(n+1)-1}{n+1}-\dfrac{2n-1}{n}=$\\

$=\dfrac{2n+1}{n+1}-\dfrac{2n-1}{n}=\dfrac{(2n+1)*n-(n+1)(2n-1)}{(n+1)*1}=$\\

$=\dfrac{n^{2}-n^{2}+1}{(n+1)*n}=\dfrac{1}{(n+1)*n} > 0$\\

Bai, beraz monotonoa da.
\end{adibide}
\end{document}