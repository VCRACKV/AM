\begin{document}

\begin{titlepage}

\begin{center}
\vspace*{-1in}
\begin{figure}[htb]
\begin{center}
\includegraphics[width=8cm]{./figuras/logo}
\end{center}
\end{figure}

ANALISI MATEMATIKOA\\
\vspace*{1.15in}
SEGIDAK. SEGIDA KONBERGENTEA\\
\vspace*{0.6in}
\begin{large}
2014-2015 IKASTURTEA\\
\vspace*{0.15in}
(EUSKERA TALDEA)\\
\end{large}
\vspace*{1.2in}
\begin{Large}
\textbf{Informatika Ingeniaritzako Gradua} \\
\end{Large}
\begin{Large}
\textbf{Informatika Fakultatea} \\
\end{Large}
\begin{Large}
\textbf{Donostia} \\
\end{Large}
\vspace*{2.3in}
\vspace*{0.3in}
\rule{80mm}{0.1mm}\\
\vspace*{0.1in}
\begin{large}
Ander Lopez \\
\end{large}
\end{center}

\end{titlepage}


\section{Segida Konbergentea}

\begin{propietate}
\item

${a_{n} }$ segida konbergentea bada, limite bakarra du.\\

\bold{Froga:}\\

Demagun ${l_{1}}$ eta ${l_{2}}$ direla  ${a_{n} }$ segidaren bi limite desberdin, ${l_{1}}\neq {l_{2}}$ beraz, $\mid{l_{1}}-{l_{2}}\mid> 0$ da.

Har dezagun $\varepsilon_{0}=\dfrac{\mid l_{1}- l_{2}\mid}{3}$ erradioa.
\vspace*{0.01in}
\begin{figure}[hbtp]
\caption{}
\centering
\includegraphics[scale=1]{../2grafikoa.jpg}
\end{figure}
\vspace*{0.01in}
Osa ditzagun $\varepsilon((l_{1},\varepsilon_{0})$ eta \varepsilon $({l_{2}},\varepsilon_{0})$ inguruneak. Ingurune horiek disjuntuak dira, hau da, \varepsilon (${l_{1}}$,\varepsilon_{0})\cap \varepsilon (${l_{2}}$, \varepsilon_{0})=\emptyset\\

-${l_{1}}$ bada ${a_{n} }$ segidaren limitea, \varepsilon_{0}>0\ emanik,\ \exists $n$ (\varepsilon_{0}) \in \NN \ / \ \forall n\geqslant n(\varepsilon_{0}) \hspace{0.75cm} ${a_{n} }$\in \varepsilon (l_{1},\varepsilon_{0})\\

-${l_{2}}$ bada ${a_{n} }$ segidaren limitea, \varepsilon_{0}>0\ emanik,\
\exists $n_{2}$(\varepsilon_{0}) \in \NN\ / \ \forall n\geqslant n_{2}(\varepsilon_{0})\\
 ${a_{n} }$\in \varepsilon (l_{2},\varepsilon_{0})\ \hspace{1cm} Hortik, n_{0} (\varepsilon_{0})=max\ \lbrace n_{1}(\varepsilon_{0}), n_{2}(\varepsilon_{0})\rbrace \ bada, \forall n\geqslant n_{0}(\varepsilon_{0})\\
 ${a_{n} }$\in \varepsilon(${l_{1}}$, \varepsilon_{0})\cap \varepsilon(${l_{2}}$, \varepsilon_{0})\ ateratzen\ dugu.\\

\\Hori ezinezkoa da inguruneak disjuntuak direlako. Ondorioz, bi limiteak ezin dira ezberdinak izan.\\
\begin{figure}[hbtp]
\caption{}
\centering
\includegraphics[scale=1]{../3grafikoa.jpg}
\end{figure}

\end{propietate}
\begin{propietate}
\item
$\{a_{n} \}$ segida konbergetea bada, bere  azpisegida guztiak konbergenteak dira eta limite bera dute.\\
\end{propietate}

\begin{propietate}
\item
${a_{n} }$ segida konbergentea bada, segida bornatua da.\\ 

{Froga:} Demagun $l$ dela ${a_{n} }$ segidaren limitea\\

$\forall \varepsilon > 0 \hspace{0.75cm}\exists n_{0}(\varepsilon)\in \NN \ / \forall n\geqslant n_{0}(\varepsilon)\hspace{0.75cm}${a_{n} }$\in \varepsilon(l,\varepsilon)\ edo \ {a_{n} }\in(l-\varepsilon, l+\varepsilon)\ edo \ (l-\varepsilon)< \ {a_{n} } \ <(l+\varepsilon)$

\ Horrek esan nahi du segidaren $\lbrace {a_{n}_{0}}, {a_{n}_{0+1}},... \rbrace$ azpimultzoa bornaturik dagoela. Bestalde, $\lbrace {a_{1} },{a_{2} },...,{a_{n}_{0-1}} \rbrace$ azpimultzoa finitua da, beraz bornatua da.
Ondorioz, $\lbrace{a_{n} }\rbrace$ segida bi azpimultzo bornatuan banatu dugu, hau da,$\lbrace{a_{n} }\rbrace$ segida ere bornatua da.\
\end{propietate}

\begin{propietate}
\item
$\lbrace{a_{n} }\rbrace$ segida konbergentearen limitea ez bada zero, segidaren gai batetik aurrera segidaren gai guztiak limitearen zeinua dute.\\
\end{propietate}
\begin{propietate}
\item
$\lbrace{a_{n} }\rbrace&$ eta $\lbrace{b_{n} }\rbrace$ segida limite $l$ bada eta $\forall n\geqslant n_{0}$ \ ${a_{n} }\leqslant{C_{n} }\leqslant{b_{n} }$ bada,$\lbrace{C_{n} }\rbrace$ segida ere konbergentea da eta bere limitea $l$ da.\\
\end{propietate}
\begin{adibide}
\item
$\lbrace sin n \rceil$ aztertuko dugu:\

$\lbrace {D_{n}} \rbrace$ = $\lbrace sin n \ / \ sin n>\dfrac{1}{2}\rbrace$ = $\lbrace 1, sin 2, sin 7,...\rbrace$ konbergentea balitz, $\dfrac{1}{2} < l_{1} <$ 1 litzateke.\\
$\lbrace {E_{n}} \rbrace$ = $\lbrace sin n \ / \ sin n<\dfrac{-1}{2}\rbrace$ = $\lbrace 4, sin 5, sin 10,...\rbrace$ konbergentea balitz, $\dfrac{-1}{2} > l_{2} >$ -1 litzateke.\\
Hortaz, ${l_{1}}$\neq${l_{2}}$ lirateke. Ondorioz $\lbrace sin n \rbrace$ ezin da konbergentea izan.\\
Dibergentea izateko $\forall k>0$ \hspace{0.3cm} $\exists n_{0}(k)\in \NN \ / \ \forall n> n_{0}(k)$ \hspace{1cm} ${a_{n} }\in \ ext (\varepsilon(0,k)) \mid{a_{n}}\mid >k$\
\begin{figure}[hbtp]
\caption{}
\centering
\includegraphics[scale=1]{../4grafiko.jpg}
\end{figure}

baina $\forall n \ sin n \in(-2,2)$, beraz, ezin da dibergentea izan.\\

Ondorioz $\lbrace sin n \rbrace$ oszilatzailea da edo $\not\exists \displaystyle{\lim_{n\to\infty} sin n}$\\
\end{adibide}
\subsection{Segida Monotonoa}
\begin{definizio}
\item
${a_{n}} \subset \RR$ segida emanik,
\begin{itemize}
\item
${a_{n}}$ monotono gorakorra da $\forall n \geqslant {n_{0}}$ \hspace{0.3cm} ${a_{n}} \leqslant {a_{n+1}}$ betetzen bada.
\item
${a_{n}}$ hertsiki monotono gorakorra da $\forall n \geqslant {n_{0}}$ \hspace{0.3cm} ${a_{n}}<{a_{n+1}}$ betetzen bada.
\item
${a_{n}}$ monotono beherakorra da $\forall n \geqslant {n_{0}}$ \hspace{0.3cm} ${a_{n+1}} \leqslant {a_{n}}$ betetzen bada.
\item
${a_{n}}$ hertsiki monotono beherakorra da $\forall n \geqslant {n_{0}}$ \hspace{0.3cm} ${a_{n+1}}<{a_{n}}$ betetzen bada.
\end{itemize}
\end{definizio}
\begin{adibide}
\item
\begin{itemize}
\item
$\lbrace 1 \rbrace$ segida monotono gorakorra eta beherakorra da.
\item
$\lbrace \dfrac{n^{2}-1}{n^{2}} \rbrace$ segida hertsiki monotono gorakorra da.
\item
$\lbrace \dfrac{n^{2}+1}{n^{2}} \rbrace$ segida hertsiki monotono beherakorra da.
\end{itemize}
\end{adibide}
\begin{teorema}
\item
Segida monotono bornatuak konbergenteak dira.
\end{teorema}
\begin{adibide}
\item
$\lbrace (1+\dfrac{1}{n})^{n} \rbrace \rightarrow e$\\

Froga daiteke edozein $n$ hartuta dagoela $2\leqslant(1+\dfrac{1}{n})^{n}<3$, hau da, bornatua dela. Segida hertsiki monotono gorakorra da.
\end{adibide}
\end{document}